\documentclass[fontsize=12pt,openright,oneside,paper=a4,BCOR=1cm]{scrbook}


\newcommand{\authorname}{Felix, Wilhelm}
\newcommand{\immatriculationnumber}{78186}
\newcommand{\studypath}{Internet Computing}
\newcommand{\currentsemester}{11}
\newcommand{\email}{wilhel40@ads.uni-passau.de}
\newcommand{\worktitle}{Prozessdesign- und Digitalisierung für eine moderne Wasserwachtverwaltung}
\newcommand{\thesistype}{Bachelorarbeit}
\newcommand{\thesisdate}{20.04.2022}
\newcommand{\thesisprof}{Prof. Harald Kosch}
\newcommand{\faculty}{Fakultät für Informatik und Mathematik}
\newcommand{\chair}{Lehrstuhl für Informatik mit Schwerpunkt Verteilte Informationssysteme}
\newcommand{\semester}{11}


%%%%%%%%%%%%%%%%%%%%%%%%%%%%%%%%%%%%%%%%%%%%%%%%%%%%%%%%%%%%

% PACKAGES:


% Use list of tabels, etc. in table of contents:
\usepackage{tocbibind}
% German paragraph skip
\usepackage{parskip}
% Encoder:????
\usepackage[utf8]{inputenc}
% Index-generation
\usepackage{makeidx}
% Einbinden von URLs:
\usepackage{url}
% Include .eps-files (needed also for the CNACC-logo):
\usepackage{epsf}
% Special \LaTex symbols (e.g. \BibTeX):
\usepackage{doc}
% Include Graphic-files:
%\usepackage{graphics}
% Include Graphic-files:
\usepackage{graphicx}
% Include doc++ generated tex-files:
%\usepackage{docxx}
% Include PDF links
\usepackage{cite}
% Include citations
\usepackage{booktabs}
\usepackage{enumerate}
% figures
%\usepackage{subcaption}

%mathstuff
\usepackage[cmex10]{amsmath}
\usepackage{amsfonts}
\usepackage{amssymb}

%hyperref for nice PDF output
\usepackage[pdftex, bookmarks=true]{hyperref}

%%%%%%%%%%%%%%%%%%%%%%%%%%%%%%%%%%%%%%%%%%%%%%%%%%%%%%%%%%%%

% OTHER SETTINGS:

% Pagestyle:
\pagestyle{headings}

% Chapter Format:
\addtokomafont{chapter}{\Large}
\RedeclareSectionCommand[%
  beforeskip=12pt,
  afterskip=3pt 
]{chapter}

% Avoid 'overhang':
\sloppy

% Choose language
\newcommand{\setlang}[1]{\selectlanguage{#1}\nonfrenchspacing}

\usepackage[german]{babel}


\setlang{german}

%%%%%%%%%%%%%%%%%%%%%%%%%%%%%%%%%%%%%%%%%%%%%%%%%%%%%%%%%%%%

% TITLE:

\begin{document}

%\thispagestyle{empty}
%\newpage

\vspace{1cm}

\begin{center}
\begin{tabular}{lr}
\includegraphics[width=6.5cm]{logouni.pdf}
\end{tabular}

\vspace{1.0cm}
\Large Universität Passau
\\
\Large \faculty
\\
\vspace{0.3cm}
\large \chair
\\
\vspace{0.3cm}
\large Betreuer: \thesisprof
\\


\end{center}


\vspace{1.5cm}

\begin{center}
        {\Huge \worktitle } % Master Thesis, Programming Project
\end{center}
\vspace{1.5cm}
\begin{center}

        {\LARGE Exposé}
        \\
        {\large
        \vspace{0.3cm}
        }
        {\Large
        zur \thesistype \\
        }
        {\large
        \vspace{0.1cm}
        \thesisdate
        }
\end{center}

\vspace{0.8cm}



\vfill {% \settowidth{\baselineskip}{0.2cm}

\vfill


{\normalsize
\begin{tabular}[l]{llll}
Name:     &  \authorname \\
Matrikelnummer:       & \immatriculationnumber \\
Fachsemester:       & \currentsemester \\
Studiengang:       & \studypath \\
E-Mail Adresse:       & \email \\
\smallskip \\

\end{tabular}}
} \cleardoublepage

%%%%%%%%%%%%%%%%%%%%%%%%%%%%%%%%%%%%%%%%%%%%%%%%%%%%%%%%%%%%


% MAIN PART:

% Inhaltsverzeichnis:
%\tableofcontents

\chapter{Einleitung}
\section{Hintergrund und Relevanz des Themas}
Die fortschreitende Digitalisierung und Automatisierung von Verwaltungsprozessen hat in den letzten Jahren auch vor gemeinnützigen Organisationen nicht haltgemacht. Insbesondere im Bereich der Wasserwacht, die sich der Rettung von Menschenleben in und um Gewässer widmet, eröffnen moderne Informationstechnologien neue Möglichkeiten, die Effizienz von Verwaltungstätigkeiten zu steigern. Im Rahmen meiner Bachelorarbeit habe ich mich mit diesem Thema auseinandergesetzt und eine Webanwendung entwickelt, um die Verwaltungsprozesse der Wasserwacht Dingolfing-Landau zu digitalisieren. 
\\
\section{Zielsetzung der Arbeit}
Das Ziel dieser Arbeit ist es, die Herausforderungen bei der Wachplanerstellung zu identifizieren und durch den Einsatz von IT-Lösungen zu bewältigen. Dabei wird insbesondere der Prozess der Wachplanung betrachtet, der eine essenzielle Rolle in der Organisation der Wasserwacht einnimmt. Traditionell werden Wachpläne manuell erstellt, was mit einem erheblichen Zeitaufwand und administrativem Aufwand verbunden ist. Durch den Einsatz einer Webanwendung sollen diese Prozesse vereinfacht und effizienter gestaltet werden, um Zeit und Ressourcen einzusparen. 
\\
\section{Methodik und Vorgehensweise}
Im Rahmen dieser Arbeit werde ich die Anforderungen an die Webanwendung analysieren, die Technologien zur Umsetzung auswählen und die Funktionalitäten der Anwendung im Detail erläutern. Dabei werden auch Aspekte wie Datensicherheit und Benutzerfreundlichkeit berücksichtigt. Die Evaluation der entwickelten Webanwendung erfolgt anhand von Nutzertests und Feedback der Wasserwachtmitglieder, um die Praxistauglichkeit und Effektivität der Anwendung zu überprüfen. 
\\
Die vorliegende Arbeit trägt somit zur Verbesserung der Verwaltungsprozesse in der Wasserwacht bei und liefert einen Beitrag zur Nutzung moderner Informationstechnologien im gemeinnützigen Bereich. Es wird erwartet, dass die entwickelte Webanwendung die Planung und Organisation von Wacheinsätzen erleichtert und somit zur Optimierung der Einsatzbereitschaft der Wasserwacht beiträgt. 
\\


%TODO
\chapter{Theoretischer Hintergrund}

\section{Wasserwacht als gemeinnützige Organisation}

\section{Verwaltungsprozesse in der Wasserwacht}

\section{Herausforderungen bei der Wachplanerstellung}

\section{Bedeutung von IT-Lösungen zur Digitalisierung von Verwaltungsprozessen} 

%TODO

\chapter{Anforderungsanalyse}
\section{Identifikation von Anforderungen an die Webanwendung} 
Bevor mit der eigentlichen Implementierung der Anwendung gestartet werden kann, müssen zunächst die Anforderungen daran festgemacht werden. \\
In einem initial bereitgestelltem Dokument (sh. Anlage 1) sind die ersten Entwürfe der Anwendung aufgezeichnet. Darin sieht man mehrere gewünschte Features, um die Verwaltungsarbeiten der Wasserwacht abzudecken, wie beispielsweise "Wachplanung", "Materialcheck", "Wachbuch" oder "Sanitätsbuch". In dieser Arbeit werden hauptsächlich die Punkte "Wachplanung" und "Wachbuch" ausgearbeitet, welche im folgenden detaillierter beschrieben werden. \\ 

\subsection{Nutzergruppen}
Für die Anwendung werden zwei Nutzergruppen angefordert. \\
Zunächst hat man die "Leitungskräfte / Admins". Diese sind für administrative Tätigkeiten zuständig und erledigen die Planung und Verwaltung der Anwendung. 
Außerdem gibt es noch die "Mitglieder / Helfer". Diese spielen vor Allem bei der Durchführung der jeweiligen Wachtage eine Rolle.

\subsection{Wachplanung}
Leitungskräften soll es ermöglicht werden im Voraus Termine für die einzelnen Wachpläne einzustellen. Diese sollen für ein konkretes Datum und Uhrzeit erstellt werden können, oder als Serie mit einer bestimmten Laufzeit. In einem Kalender sollen die einzelnen Termine dargestellt werden, mit einer farblichen Kennung, ob für diesen Termin bereits Mitarbeiter eingebucht sind. \\
Für Mitglieder ist es notwendig, sich für Wachtage einzubuchen. Dazu soll auch der Kalender genutzt werden, um eine Übersicht der verfügbaren Wachtage zu erhalten.

\subsection{Wachbuch}
Im Wachbuch werden die Ereignisse eines Wachtages festgehalten. Nutzern soll es möglich sein, eine Liste der gebuchten Helfern anzuzeigen, sowie nach Wachstart eine Liste der anwesenden Helfern anzuzeigen. Nutzer können sich für diesen Wachtag ein-, bzw. ausbuchen. \\
Weiterhin werden für diesen Wachtag die vorliegenden Wetterdaten vorliegen, welche über eine API ermittelt werden. Nutzer können die aktuelle Wassertemperatur über ein Eingabefeld abspeichern. \\
Die Ereignisse des Wachtags werden in einer Tabelle, mit Namen und Zeitstempel, festgehalten. Dabei werden der Wachbeginn und das Wachende der Mitglieder, die An- und Abmeldung im ILS, die automatischen Wetterdaten und der Wachstart/ -ende automatisch eingetragen. Nutzer können zusätzlich manuell Eingaben tätigen. \\
Leitungskräften soll es zusätzlich ermöglicht werden eine Auswertung eines Wachtages vorzunehmen. Dazu gibt es einen Button "Drucken", über den ein PDF, welches die Ereignisse des Wachtages zusammenfasst, heruntergeladen werden kann. Die Informationen, welche dieses Dokument enthalten muss, wurden aus einem derzeit verwendeten Wachbuch übernommen (sh. Anlage 2). Darin befinden sich die vorliegenden Wetterdaten, Informationen über das anwesende Wachpersonal, sowie das Wachtagebuch. \\
Die Punkte "Übersicht Checks" und "Einsätze" wurden im Rahmen dieser Arbeit nicht betrachtet. 

\section{Befragung von Wasserwachtmitgliedern und Mockups}
In mehreren Iterationen wurden mit Herrn Andreas Schmeisl (Vorsitzender Kreiswasserwacht Dingolfing-Landau) und Herrn Werner Gerl (Technischer Leiter) die finalen Vorgaben geklärt. Dabei wurde mit Hilfe von Mockups ein Zielbild der Anwendung entwickelt. 
\subsection{Mockups}
\subsubsection{Was sind Mockups}
%Definition Mockups, warum werden sie in der Planung von Software eingesetzt
Mockups werden in der Softwareentwicklung eingesetzt um sicher zu gehen, dass man die Anforderungen an die Anwendung verstanden hat. Mockups haben den Vorteil, dass sie kosten- und zeiteffizient erstellt werden können und schnell anpassbar sind.

\subsubsection{Papier-Mockups}
In einem initialen Meeting wurden mit Papier Mockups die groben Umrisse der Website aufgezeigt. Es wurden außerdem noch weitere Funktionalitäten und Anforderungen besprochen, die aus den vorliegenden Unterlagen nicht hervorgingen. \\
Mit den Ergebnissen aus diesem Meeting konnte ein Prototyp der Anwendung erstellt werden.


\subsubsection{Figma}
%Website beschreiben 
Figma ist eine Website, über die man Protoypen und Mockups erstellen kann. Über einzelne Boards können die verschiedenen Seiten der Zielanwendung simuliert werden. Der Vorteil hierbei ist, dass auch Nutzereingaben simuliert werden können. So kann beispielsweise bei Klicken ein Seitensprung animiert werden. Dadurch kann man ohne viel Aufwand dem Kunden zeigen, wie die Anwendung einmal aussehen wird. 

\subsection{Ergebnisse}
%Bilder der Mocks darstellen, einzelne Masken wurden konkretisiert, weitere Anforderungen wurden aufgenommen, Emailversand bei Login. 
Mit Figma wurden alle vorgesehenen Seiten der Website dargestellt. Dabei kamen noch weitere Anforderungen auf: \\
Beim Login Prozess wurde gefordert, dass die "Admins" bei einer Neu-Anmeldung über Email benachrichtigt werden. Neue Nutzer können daher nach dem registrieren nicht sofort auf die Seite zugreifen, sondern müssen erst warten bis ein Admin sie bestätigt hat.\\



Admins sehen auf der Nutzerübersicht alle Nutzer und ob sie bereits freigeschaltet wurden. Über das jeweilige Nutzerprofil können sie den Nutzer über einen Button freischalten. Danach wird der Nutzer per Email benachrichtigt und erhält Zugang zum System. \\


Bei der Wachplanung wurde angefordert, dass Wachtermine einzeln für ein bestimmtes Datum und Uhrzeit angelegt werden können. Außerdem soll es möglich sein gleich eine Reihe an Terminen anzulegen. Dazu muss ein Start- und Enddatum angegeben werden, die Uhrzeit, sowie die Wochentage an denen ein Termin eingestellt werden soll. Als Übersicht für die bisherigen Termine wird ein Kalender verwendet. Hier wurde überlegt, ob man die einzelnen Termine farbig codiert, wenn beispielsweise ein "Rettungsschwimmer" und "Wachleiter" noch nicht eingebucht sind. In weiteren Abstimmungsterminen hat man sich aber darauf geeinigt, die Termine rot einzufärben, wenn noch niemand eingebucht ist, und grün einzufärben wenn Nutzer eingebucht sind.



\renewcommand{\cleardoublepage}{}
\chapter{Motivation}
Aus einer rein analogen Bewältigung von Verwaltungsprozessen ergeben sich eine Reihe von Nachteilen. Einerseits hat man ein Problem mit fehlender Datensicherung. Da man nur ein Dokument mit den benötigten Informationen hat, könnte dies zu Problemen führen wenn dieses beschädigt wird oder verloren geht. Wenn die Daten allerdings in einer Datenbank persistiert werden, kann man leicht ein Backup erstellen. \\
Des weiteren ist man nicht flexibel bezüglich des Informationsflusses. Eine digitale Repräsentation in einer Webanwendung erlaubt es den Nutzern sich einen besseren und schnelleren Überblick zu erschaffen, da man nicht vor Ort sein muss, um die Informationen abzufragen. Mit dieser Plattform soll eine leichtere Abstimmungsmöglichkeit geboten werden innerhalb der Wasserwacht. 

\renewcommand{\cleardoublepage}{}
\chapter{Related Work}
Da die Webanwendung von unterschiedlichen Endgeräten aufgerufen werden kann, muss darauf geachtet werden, diese auch mit unterschiedlichen Browser-Auflösungen zu testen (vgl. \cite[S. 49]{sklar2011principles}). Nur so kann eine größtmögliche Flexibilität gewährleistet werden. \\
Ein weiterer interessanter Aspekt bei der Entwicklung ist die Barrierefreiheit. Hierbei wird darauf geachtet, vor allem Menschen mit Behinderung  die Möglichkeit zu bieten, die Website \glqq uneingeschränkt nutzen zu können\grqq{} \cite[S. 2]{probiesch2013barrierefreiheit}. Es gibt verschiedene Technologien, wie zum Beispiel \glqq Screen - und Webreader\grqq{} oder \glqq Vergrößerungssysteme\grqq{} \cite[S. 3]{probiesch2013barrierefreiheit}, die es ermöglichen diese Barrieren zu überwinden.


\renewcommand{\cleardoublepage}{}
\chapter{Metodologie}
Das Vorgehen in dieser Arbeit wird in Zusammenarbeit mit der Wasserwacht erfolgen. \\
In einem initialem Treffen mit der Wasserwacht, sollen die notwendigen Prozessschritte konkretisiert werden, um die Anforderungen an die Anwendung auszumachen und die Prozesse zu modellieren. Wenn diese feststehen, wird in weiteren Abstimmungsterminen mit Mockups versucht die richtige Designentscheidung zu treffen. Danach kann mit der Implementierung der Web-Anwendung begonnen werden. \\
Plan ist es, mit dem Spring Framework eine Java-Webanwendung aufzubauen, mit Postgres Datenbank Anbindung. Als Frontend-Framework wird Thymeleaf verwendet. \\
Zur abschließenden Evaluierung wird in einem Experten-Interview überprüft, zu welchem Grad die umgesetzte Anwendung mit den Anforderungen der Wasserwacht übereinstimmt.

\renewcommand{\cleardoublepage}{}
\chapter{Zeitplan}
Ein genauer Zeitplan kann noch nicht festgemacht werden, da die konkreten Anforderungen erst in Abstimmungsterminen mit der Wasserwacht entstehen. Hier ist noch nicht klar wie viele Iterationen, v.A. im Hinblick auf Designfindung, benötigt werden. Erst wenn diese Punkte geklärt wurden, kann mit der Implementierung begonnen werden. 

% References (Literaturverzeichnis):
\bibliographystyle{alpha}
\bibliography{Literatur}


%%%%%%%%%%%%%%%%%%%%%%%%%%%%%%%%%%%%%%%%%%%%%%%%%%%%%%%%%%%%
\end{document}
